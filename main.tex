\documentclass{article}
\begin{document}
    \textbf{Threads:} Ejecucion de pequeños pasos sistematicos que realiza un tarea. Estados de los procesos:
    \begin{itemize}
        \item Nuevo
        \item Listo
        \item Ejecucion O Bloqueo       
        \item Terminado
    \end{itemize}   
    \textbf{Process Control Block (PCB):} Estructura de datos que contiene toda la informacion necesaria para gestionar el proceso. Los elementos tipicos de un PCB incluyen:
    \begin{itemize}
        \item\textbf{Identificador de Proceso (PID):} Un número único que identifica al proceso.  
        \item\textbf{Estado del Proceso:} Indica el estado actual del proceso (listo, en ejecución)
        \item\textbf{Contador de programa:}
        \item\textbf{Registros de CPU}
    \end{itemize}

    \textbf{Maquinas Virtuales:} Software que emula el hardware de un computador, permitiendo ejecutar diferentes OS
    \begin{enumerate}
        \item\textbf{Beneficio:} Tener diferentes OS en un solo computador.
        \item\textbf{Hypervisor:} Herramienta o Software que gestiona las maquinas Virtuales
        \item\textbf{Por que usar:}
            \begin{itemize}
                \item Flexibilidad
                \item Seguridad
                \item Aprendizaje avanzado
                \item Crecimiento profesional
            \end{itemize}
        \item\textbf{Requisitos: }
            \begin{itemize}
                \item Minimo 4GB RAM
                \item OS
                \item Software (Gestor de máquinas virtuales)
            \end{itemize}
    \end{enumerate}
\end{document}

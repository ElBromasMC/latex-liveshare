\documentclass[12pt, a4paper]{article}
\usepackage{graphicx}
\usepackage[utf8]{inputenc}
\usepackage[spanish]{babel}
\usepackage[left=2.54cm, right=2.54cm, top=2.54cm, bottom=2.54cm]{geometry}
\usepackage{background}
\usepackage{enumitem}
\usepackage{amsmath}
\usepackage{amssymb}
\usepackage{amsthm}
\usepackage{float}
\usepackage{hyperref}
\usepackage{xcolor}
\usepackage{color, colortbl}
\usepackage{paracol}
\usepackage{multicol}
\usepackage{siunitx}
\usepackage{multirow}
\usepackage{booktabs}
\usepackage{array}
\usepackage{setspace}
\usepackage{fancyhdr}
\pagestyle{fancy}
\usepackage{url}
\usepackage{relsize} 
\begin{document}
\backgroundsetup{
        placement = center,
        angle = 0,
        scale = 1,
        opacity = 0.8,
        %contents = {\includegraphics[scale = 1]{img/fondo_san_marcos.jpg}}}
}

\begin{titlepage}
\begin{center}
{ \LARGE\textbf{Universidad Nacional Mayor de San Marcos}}\\
 \vspace{3.5mm}
{\large \textbf{Decana de América, Universidad del Perù}}\\
  \vspace{3.5mm}
  \begin{figure}[h]
      \centering
      \includegraphics[height=8.5cm]{img/800px-UNMSM_escudo_XVI-XXI_transparente_siglas_nombre_vertical.svg.png}
  \end{figure}
{ \Large\textbf{Facultad de Ciencias Matematicas}}\\
 \vspace{3.5mm}
{ \large\textbf{Escuela Profesional de Computación Científica}}\\
 \vspace{3.5mm}
{ \large\textbf{Semestre - 2024 I}}\\
{\rule{\linewidth}{0.55mm}} 
 { \Large\textbf{Ecuaciones Diferenciales Ordinarias\\ Ejercicios}}\\
{\rule{\linewidth}{0.55mm}} 
{ \large\textbf{Docente:Jorge Icaro Condado Jauregui}}\\ \vspace{3.5mm}
{ \large\textbf{Integrantes: }}\\
 \vspace{3.5mm}
\begin{itemize}
    \item Tanaka Matheus, Louiggiiiiiiiii
    \item Vilchez Quispe, Yoshiro Cardichhhhh
    \item Porras Anco, Sebastian Aaronnnnnnnnn
    \item Espinoza Huaman, Diego Alexhander 
    \item Solimano Cure, Franco Daviddddd 
    \item Linares Rojas, Ander Rafael
    \item Madrid Llanos, Karla Patricia 
\end{itemize}
\vfill
{ \huge\textbf{2024}}
\end{center}
\end{titlepage}
\backgroundsetup{
        placement = center,
        angle = 0,
        scale = 1,
        contents = {}}
\newpage
si carga, ya testee varias cosas xd \\
por eso estoy usando lualatex como compilador, el
pdflatex se bugeaba
ya mira si accedes a localhost:8000/build/main2.pdf puedes ver el otro documento
solo es un servidor http de python asi que tambien puedes ver todos los archivos
si ingresas a localhost:8000/build
y creo que va mucho mas rapido que
el de overleaf :v aunque mas misio

\end{document}
